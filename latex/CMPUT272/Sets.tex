\documentclass{report}
\usepackage{amsthm}
\usepackage{amssymb}

\title{Title}
\author{Julian Pagcaliwagan}
\date{}

\theoremstyle{definition}
\newtheorem{definition}{Definition}

\begin{document}
    \maketitle
    \section*{Sets}
    \subsection*{Definitions}
        \begin{definition}[Sets]
            A set is an \textbf{unordered} collection of objects called elements, denoted by $ \left\{ \right\} $
        \end{definition}

        \begin{definition}[Classifications of numbers]
            We can categorize classifications of numbers as sets. We have
            \begin{itemize}
                \item $ \mathbb{Z} = \left\{ -2, -1, 0, 1, 2, ... \right\} $
                \item $ \mathbb{N} = \left\{ 0, 1, 2, 3, ... \right\} $
                \item $ \varnothing = \left\{  \right\} $
                \item $ \mathbb{Q} = \left\{ \frac{a}{b}: a, b \in \mathbb{N}, b \neq 0 \right\} $
            \end{itemize}
        \end{definition}

        \noindent The last item, the rationals calls to mind set builder notation, where sets can be built with conditions.
        We can invoke set-builder notation: 
        \begin{center}
            \{elements : conditions used to generate the elements\}
        \end{center}

        Since math is a logical and interpretable language, we can interpret the set-builder notation of the rationals to be
        \[
            \mathbb{Q} = \left\{ \frac{a}{b}: a, b \in \mathbb{N}, b \neq 0 \right\}
        \]
        The set of all rational numbers is defined to be the set of all fractions in the form $\frac{a}{b}$ such that $a$ and $ b $ are integers and $ b $ is nonzero.

    \subsection*{Proving $ A \subseteq B $}
    
    
\end{document}