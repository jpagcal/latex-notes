\documentclass{report}
\usepackage{amsthm}
\usepackage{amssymb}
\usepackage{amsmath}

\title{Lines}
\author{Julian Pagcaliwagan}
\date{December 13, 2022}

\theoremstyle{definition}
\newtheorem{definition}{Definition}

\begin{document}
\maketitle
\section*{Lines in $ \mathbb{R}^n $}
    \subsection*{The Equation of a Line}
        \noindent We all know intuitively that a line is represented by the equation $ y = mx + b $, but a line can also be represented using a pair of two vectors
        \begin{definition}[Equation of a line]
            let $ \mathbf{d}$ be the direction vector abd $ \mathbf{p} $ be the position vector, then we can represent a line as 
            the position vector plus a scalar multiple of the direction vector
            \[
                l = \left\{ \mathbf{p} + t \mathbf{d} | t \in\mathbb{R} \right\}
            \]
            where $ l $ is a line that passes through the point $ \mathbf{p} $ and is parallel to the direction vector $ \mathbf{d} $
            Given the set-builder notation on the definition of a line, we should consider a line to be the \textbf{set of all position vectors $ \overrightarrow{OX}$ to points on the line}\\
            Similarly, we can view a line as a function corresponding to the scalar multiple $ t $
            \[
                \mathbf{x(t)} = \mathbf{p} + t \mathbf{d}
            \]
        
            \noindent Note that this implies that, given two position vectors, we can conclude that the direction vector is the distance between those two
            \[
                \mathbf{x(t)} = \mathbf{ \mathbf{p_1}} + t ( \mathbf{ \mathbf{p_1}} - \mathbf{ \mathbf{p_1}})
            \]
            \noindent a corollary of this definition is that the same lines can be represented using different position and direction vectors. in this case, $ \mathbf{p} $ can take different points on the line $ l $, or $ \mathbf{d} $ can take any scalar multiple of the original direction vecctor.

        \end{definition}
    \subsection*{Intersections of two lines}
        The intersection of two lines is given by the system of equations
        \[
            x_1 = x_2
        \]
        where $ x_1 $ is the vector at the intersection point and $ x_2 $ is also the vector at the intersection point
    \subsection*{General Equation of a line}
        The general equation of a line is given by
        \[
            ax_1 + by_1 = c
        \]
        where $ x_1 $ and $ y_1 $ correspond to some arbitrary point on the line
        and the vector $ \left[ a \ b \right] $  is a vector orthogonal to the line. This implies that 
        \[
            \left[ a \ b \right] \cdot \left[ x_2 - x_1 \ y_2 - y_1 \right] = 0 
        \]   
        \noindent Explicitly, the equation of a line in normal form is represented by 
        \[
            \mathbf{n} \cdot \mathbf{x} = \mathbf{n} \cdot \mathbf{p} 
        \]

\end{document}