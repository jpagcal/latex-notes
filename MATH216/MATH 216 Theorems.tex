\documentclass{report}
\usepackage{amsthm}
\usepackage{amssymb}
\usepackage{enumerate}
\usepackage{amsmath}
\usepackage{mdframed}

\title{Math 216 Theorems}
\author{Julian Pagcaliwagan}
\date{September 7, 2023}

\theoremstyle{definition}
\newtheorem{definition}{Definition}
\newtheorem{lem}{Lemma}
\newtheorem{coro}{Corollary}
\newtheorem{prop}{Proposition}

\begin{document}
\maketitle

\newtheorem{t1}{Theorem}


\section*{III. Sequences in $\mathbb{R}$}
\newtheorem{seqt}{Theorem}
\newtheorem{seql}[seqt]{Lemma}
\newtheorem{seqc}[seqt]{Corollary}
\newtheorem{seqp}[seqt]{Proposition}

\begin{seql} Let $(a_n)$ be a sequence.
    \begin{enumerate}[(i)]
        \item $(a_n)$ can have at most one limit.
        \item If $(a_n)$ is convergent, then $(a_n)$ is bounded.
    \end{enumerate}
\end{seql}

\begin{seqt}
    Let $(a_n), (b_n)$ be sequences in $\mathbb{R}$ such that $(a_n) \rightarrow a \text{ and } (b_n) \rightarrow b$. Then
        \[ \lim_{n\to\infty} (a_n \pm b_n) = \lim_{n\to\infty} (a_n) \pm \lim_{n\to\infty} (b_n)\] 
        \[ \lim_{n\to\infty} (a_n) \cdot (b_n) = \lim_{n\to\infty} (a_n) \cdot \lim_{n\to\infty} (b_n)\]
        \[ \lim_{n\to\infty} \frac{(a_n)}{b_n} = \frac{\lim_{n\to\infty} (a_n)}{\lim_{n\to\infty} (b_n)}\]
        \[ \lim_{n\to\infty} |(a_n)| = |\lim_{n\to\infty} (a_n)|\]
\end{seqt}

\begin{seqt}
    Let $(a_n), (b_n)$ be sequences in $\mathbb{R}$ such that $(a_n) \rightarrow a \text{ and } (b_n) \rightarrow b$. Assume now that
    $\forall n\in\mathbb{N}, a_n\leq b_n.$ Then
    \[ \lim_{n\to\infty} (a_n) \leq \lim_{n\to\infty} (b_n)\]
\end{seqt}

\begin{seqt}
    Let $(a_n), (b_n), (c_n)\text{ be sequences in }\mathbb{R}.$ If $(a_n)\to a, (c_n)\to a, \text{ and } \forall n\in\mathbb{N}, a_n \leq b_n \leq c_n$,
    then $(b_n)\to a$. 
\end{seqt}

\begin{seqt}
    For every monotone sequence $(a_n) \text{ in } \mathbb{R}$, the following are equivalent:
    \begin{enumerate}[(i)]
        \item $(a_n)$ is convergent
        \item $(a_n)$ is bounded
    \end{enumerate}
\end{seqt}

\begin{seqt}
    Let $I_1 \supset I_2 \supset I_3 \supset \dots$ be non-empty closed intervals. Then the set
    \[\bigcap_{n\in\mathbb{N}} I_n := \{c\in\mathbb{R}:c\in I_n\} \]
    is non-empty.
\end{seqt}

\begin{seql}
    Assume that $A_j$ is a countable set $\forall j\in\mathbb{N}$. Then
    \[\bigcup_{j\in\mathbb{N}} A_j := \left\{ a: a\in A_j \text{ for some } j\in\mathbb{N} \right\}\]
    is also countable.
\end{seql}

\begin{seqt}
    $\mathbb{Q}$ is countable.
\end{seqt}

\begin{seqt}
    $\mathbb{R}$ is uncountable.
\end{seqt}

\begin{seql}
    Let $(a_n)$ be a sequence in $\mathbb{R}$. If $(a_n)$ converges to $a\in\mathbb{R}$ then any subsequence $(a_{n_j})$ of $(a_n)$
    also converges to $a$. i.e. $\mathbb{S}[a_n] = \{a\}$
\end{seql}

\begin{seql}
    Every sequence $(a_n)$ in $\mathbb{R}$ has a monotone subsequence.
\end{seql}

\begin{seqt}
    Every bounded sequence in $\mathbb{R}$ has a convergent subsequence.
\end{seqt}

\begin{seql}
    Let $(a_n)$ be a sequence in $\mathbb{R}$. Then
    \begin{enumerate}[(i)]
        \item  $\mathbb{S}[a_n]\neq\varnothing$
        \item $\mathbb{S}[a_n] \subset \mathbb{R}\iff (a_n)$ is bounded.
        \item $\mathbb{S}[a_n] = \{a\}\iff\lim_{n\to\infty}(a_n)=a$
    \end{enumerate}
\end{seql}



\end{document}